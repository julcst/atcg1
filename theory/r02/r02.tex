\documentclass{article}
\usepackage{graphicx}
\usepackage[left=3.5cm, right = 3.5cm, top=3.5cm, bottom=3.5cm, head=13.6pt]{geometry}
\usepackage[onehalfspacing]{setspace}
\usepackage{amsthm}
\usepackage{amsmath}
\usepackage{amssymb}
\usepackage{mathtools}
\usepackage{float}
\usepackage{algpseudocode}
\usepackage{algorithm}
\usepackage{comment}
\usepackage{csquotes}


\title{Advanced Topics in Computer Graphics I - Sheet R02}
\author{Ninian Kaspers, Robin Landsgesell, Julian Stamm}
\date{\today}

\begin{document}

\maketitle

\section*{Assignment 2}

\subsection*{Ray-Sphere intersection}

Given a ray $r(t)$ with origin $o$ and direction $d$:
\begin{equation*}
  r(t) = o + t \cdot d  
\end{equation*}
and a sphere with center $c$ and radius $r$ defined by:
\begin{equation*}
  \Vert x - c \Vert^2 = r^2
\end{equation*}
Substitute ray equation into sphere equation:
\begin{align*}
  \Vert o + t \cdot d - c \Vert^2 &= r^2\\
  \Vert o + t \cdot d - c \Vert^2 - r^2 &= 0 &\vert\text{ Substitute }o-c=s\\
  (s + t \cdot d) \cdot (s + t \cdot d) - r^2 &= 0\\
  s \cdot s + t \cdot d \cdot s + t \cdot d \cdot s + t \cdot d \cdot t \cdot d - r^2 &= 0\\
  d \cdot d \cdot t^2 + 2 \cdot d \cdot s \cdot t + s \cdot s - r^2 &= 0 &\vert\text{ Substitute back }o-c=s\\
  d \cdot d \cdot t^2 + 2 \cdot d \cdot (o-c) \cdot t + (o-c) \cdot (o-c) - r^2 &= 0
\end{align*}
This results in a quadratic equation of the form $at^2 + bt + c = 0$ with
\begin{align*}
  a &= d \cdot d\\
  b &= 2 \cdot d \cdot (o-c)\\
  c &= (o-c) \cdot (o-c) - r^2
\end{align*}
The solutions of this are:
\begin{equation*}
  t = \frac{-b \pm \sqrt{b^2 - 4ac}}{2a}
\end{equation*}

\section*{Assignment 3}

\subsection*{Ray-Triangle intersection}
Given a ray $r(t)$ with origin $o$ and direction $d$:
\begin{equation*}
  r(t) = o + t \cdot d
\end{equation*}
and a triangle with vertices $a$, $b$, $c$ in global coordinates.
\subsection*{ Barycentric Coordinates }


Precompute normal n of triangle plane:
        \begin{equation*}
            n = (b-a) \times (c-a)
        \end{equation*}
    and value $p = - n \cdot a$ for all triangles.
\begin{enumerate}
    \item \textbf{if } $n \cdot d$ = 0: \textbf{return false } (3 mult, 2 add)
    \item Compute parameter $t$: (3 mult, 3 add, 1 div)
        \begin{equation*}
            t = -\frac{p + (n \cdot o)}{n \cdot d}
        \end{equation*}
    \item \textbf{if} $t \leq 0$: \textbf{return false }
    \item Compute intersection point $q$: (3 mult, 3 add)
        \begin{equation*}
            q = o + t \cdot d
        \end{equation*}
    \item Compute barycentric coordinates $\alpha, \beta, \gamma$ of intersection point $q$ with respect to the triangle vertices: (27 mult, 17 add/sub, 2 div)
        \begin{equation*}
                \alpha = \frac{((b-c) \times (q-c))\cdot n}{((b-a) \times (c-a)) \cdot n}
        \end{equation*}
        \begin{equation*}
                \beta = \frac{((c-a) \times (q-a))\cdot n}{((b-a) \times (c-a)) \cdot n}
        \end{equation*}
         \begin{equation*}
                \gamma = 1 - \alpha - \beta
        \end{equation*}
    \item \textbf{if }$\alpha \geq 0, \beta \geq 0, \gamma \geq 0$: \textbf{return true}
\end{enumerate}

Total Operations with $n$ and $p$ precomputed:\\
Multiplications: 36\\
Additions/Subtractions: 25\\
Divisions: 3


\subsection*{Möller-Trumbore}
\begin{enumerate}
    \item Compute Edges $e1$, $e2$: (6 sub)
            \begin{equation*}
                e_1 = b - a
            \end{equation*}
                        \begin{equation*}
                e_2 = c - a
            \end{equation*}
    \item Compute Cross Product $h$: (6 mult, 3 sub)
            \begin{equation*}
                h = d \times e_2
            \end{equation*}
    \item Compute the determinant: (3 mult, 2 add)
        \begin{equation*}
            \text{det} = e_1 \cdot h
        \end{equation*}
    \item \textbf{if } $|\text{det}| < \epsilon$: \textbf{return false} (ray is parallel to triangle plane)
    \item Compute inverse determinant: (1 div)
        \begin{equation*}
            \text{det\_inv} = \frac{1}{\text{det}}
        \end{equation*}
    \item Compute vector $s$ from vertex $a$ to ray origin: (3 sub)
        \begin{equation*}
            s = o - a
        \end{equation*}
    \item Compute first barycentric coordinate $u$: (4 mult, 2 add)
        \begin{equation*}
            u = (s \cdot h) \times \text{det\_inv}
        \end{equation*}
    \item \textbf{if } $u < 0$ \textbf{or} $u > 1$ : \textbf{return false}
    \item Compute cross product $q$: (6 mult, 3 sub)
    \begin{equation*}
        q = s \times e_1
    \end{equation*}
    \item Compute second barycentric coordinate  $v$:  (4 mult, 2 add)
        \begin{equation*}
            v = (d \cdot q) \times \text{det\_inv}
        \end{equation*}
    \item \textbf{if } $v < 0$ \textbf{or} $u+v > 1$ : \textbf{return false}
    \item Compute parameter $t$ and intersection point $q$ (7 mult, 5 add):
        \begin{equation*}
            t = (e_2 \cdot q) \times \text{det\_inv}
        \end{equation*}
        \begin{equation*}
                q = o + t \cdot d
        \end{equation*}
\end{enumerate}

Total Operations: \\
Multiplications: 30\\
Additions/Subtractions: 26\\
Divisions: 1\\
\\
In the case of a successful intersection, the Möller–Trumbore algorithm requires six fewer multiplications, 2 fewer (costly) divisions and one additional addition, assuming sequential execution without parallelization.\\
Fewer divisions and cross product calculations also lead to improved numerical stability compared to the approach based on barycentric coordinates.
\end{document}