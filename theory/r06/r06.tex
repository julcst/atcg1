\documentclass{article}
\usepackage{graphicx}
\usepackage[left=3.5cm, right = 3.5cm, top=3.5cm, bottom=3.5cm, head=13.6pt]{geometry}
\usepackage[onehalfspacing]{setspace}
\usepackage{amsthm}
\usepackage{amsmath}
\usepackage{amssymb}
\usepackage{mathtools}
\usepackage{float}
\usepackage{algpseudocode}
\usepackage{algorithm}
\usepackage{comment}
\usepackage{csquotes}
\usepackage{enumitem}


\title{Advanced Topics in Computer Graphics I - Sheet R06}
\author{Ninian Kaspers, Robin Landsgesell, Julian Stamm}
\date{\today}

\begin{document}

    \maketitle

    \section*{Assignment 2}

    In this exercise you should manually determine the following integral using various methods:

    \begin{align*}
        I &= \int_{-\frac{\pi}{2}}^{\frac{\pi}{2}} e^{-x^2} \, dx
    \end{align*}

    % When random numbers are needed, use the following 20 numbers sampled uniformly from the interval [0, . . . , 1]:
    % 0.4387 0.7655 0.1869 0.4456 0.7094 0.2760 0.6551 0.1190 0.9597 0.5853
    % 0.3816 0.7952 0.4898 0.6463 0.7547 0.6797 0.1626 0.4984 0.3404 0.2238
    % as numpy array R = np.array([0.4387, 0.7655, 0.1869, 0.4456, 0.7094, 0.2760, 0.6551, 0.1190, 0.9597, 0.5853, 0.3816, 0.7952, 0.4898, 0.6463, 0.7547, 0.6797, 0.1626, 0.4984, 0.3404, 0.2238])
    
    a) Calculate the integral numerically with a tool of your choice (maple, wolfram alpha, ...).

    \begin{align*}
        I &= \sqrt{\pi} \text{erf}(\frac{\pi}{2}) \approx 1.7258
    \end{align*}

    b) Calculate the integral using naive Monte Carlo Integration (samples drawn from a constant PDF). Specify PDF, CDF, inverse CDF, and the final result.

    The PDF is given by $p(x) = \frac{1}{\pi}$ for $x \in [-\frac{\pi}{2}, \frac{\pi}{2}]$ and 0 otherwise.
    
    The CDF is given by $F(x) = \frac{x + \frac{\pi}{2}}{\pi}$ for $x \in [-\frac{\pi}{2}, \frac{\pi}{2}]$.
    
    The inverse CDF is given by $F^{-1}(u) = u \cdot \pi - \frac{\pi}{2}$ for $u \in [0, 1]$.

    The final result using the 20 random numbers provided is:

    \begin{align*}
        I = \frac{1}{20} \sum_{x \in R} \frac{e^{-F^{-1}(x)^2}}{p(x)} \approx 2.0553025348193215
    \end{align*}

    c) Calculate the integral using Monte Carlo Integration B and Importance Sampling. Draw the necessary samples from a probability distribution proportional to g(x) = cos (x). As in the previous task, specify PDF, CDF, inverse CDF, and the final result.

    The PDF is given by $p(x) = \frac{1}{\pi} \cos(x)$ for $x \in [-\frac{\pi}{2}, \frac{\pi}{2}]$ and 0 otherwise.

    The CDF is given by $F(x) = \frac{\sin(x)}{\sin(\frac{\pi}{2})} = \sin(x)$ for $x \in [-\frac{\pi}{2}, \frac{\pi}{2}]$.

    The inverse CDF is given by $F^{-1}(u) = \arcsin(u)$ for $u \in [0, 1]$.

    The final result using the 20 random numbers provided is:

    \begin{align*}
        I = \frac{1}{20} \sum_{x \in R} \frac{e^{-F^{-1}(x)^2}}{p(x)} \approx 2.564792100610645
    \end{align*}

    d) 
    Variance in b) is:
    \begin{align*}
        \text{Var}(I) \approx 0.6845154176519112
    \end{align*}

    Variance in c) is:
    \begin{align*}
        \text{Var}(I) \approx 0.4800330782913318
    \end{align*}

    Variance in c) is lower than in b) because the samples are drawn from a distribution that is more proportional to the function we are integrating, thus reducing the variance.

    e) Because of its high dimensionality other methods of integration suffer from curse of dimensionality. Monte Carlo not.

\end{document}
