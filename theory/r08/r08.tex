\documentclass{article}
\usepackage{graphicx}
\usepackage[left=3.5cm, right = 3.5cm, top=3.5cm, bottom=3.5cm, head=13.6pt]{geometry}
\usepackage[onehalfspacing]{setspace}
\usepackage{amsthm}
\usepackage{amsmath}
\usepackage{amssymb}
\usepackage{mathtools}
\usepackage{float}
\usepackage{algpseudocode}
\usepackage{algorithm}
\usepackage{comment}
\usepackage{csquotes}
\usepackage{enumitem}
\usepackage{stmaryrd}

\title{Advanced Topics in Computer Graphics I - Sheet R08}
\author{Ninian Kaspers, Robin Landsgesell, Julian Stamm}
\date{\today}

\begin{document}

    \maketitle

    \section*{Assignment 2}

    a) Photon Mapping does not converge to the correct result when the sample count approaches infinity.

    b) By decreasing the gathering radius towards zero, the bias converges to zero.
    Thus, Photon Mapping is a consistent estimator.

    c) The Photons are stored in a spatial tree structure like a k-d-tree or a BVH.
    Thus, we can solve fixed radius nearest neighbor queries efficiently in logarithmic time.

    d) No, we do not have to regenerate the photon map when changing the scene, as the photons are stored with respect to their global position.
    We only have to rebuild when the scene is changed.

    e) As indirect illumination is generally low frequency, we can use a low number of photons and a large radius to achieve a good result.
    Caustics, on the other hand, tend to be high frequency and require a large number of photons and a small radius to achieve a good result.
    Thus, we can split the photons in a caustic and a global map by storing if a specular sample occured between the photon and the light source.
    We then can use different radii for the two maps, which allows us to get rid of high frequency noise in the global map while simultaneously achieving high quality caustics.
\end{document}

